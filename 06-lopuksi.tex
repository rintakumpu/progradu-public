% Options for packages loaded elsewhere
\PassOptionsToPackage{unicode}{hyperref}
\PassOptionsToPackage{hyphens}{url}
%
\documentclass[
]{article}
\usepackage{amsmath,amssymb}
\usepackage{iftex}
\ifPDFTeX
  \usepackage[T1]{fontenc}
  \usepackage[utf8]{inputenc}
  \usepackage{textcomp} % provide euro and other symbols
\else % if luatex or xetex
  \usepackage{unicode-math} % this also loads fontspec
  \defaultfontfeatures{Scale=MatchLowercase}
  \defaultfontfeatures[\rmfamily]{Ligatures=TeX,Scale=1}
\fi
\usepackage{lmodern}
\ifPDFTeX\else
  % xetex/luatex font selection
\fi
% Use upquote if available, for straight quotes in verbatim environments
\IfFileExists{upquote.sty}{\usepackage{upquote}}{}
\IfFileExists{microtype.sty}{% use microtype if available
  \usepackage[]{microtype}
  \UseMicrotypeSet[protrusion]{basicmath} % disable protrusion for tt fonts
}{}
\makeatletter
\@ifundefined{KOMAClassName}{% if non-KOMA class
  \IfFileExists{parskip.sty}{%
    \usepackage{parskip}
  }{% else
    \setlength{\parindent}{0pt}
    \setlength{\parskip}{6pt plus 2pt minus 1pt}}
}{% if KOMA class
  \KOMAoptions{parskip=half}}
\makeatother
\usepackage{xcolor}
\usepackage[margin=1in]{geometry}
\usepackage{longtable,booktabs,array}
\usepackage{calc} % for calculating minipage widths
% Correct order of tables after \paragraph or \subparagraph
\usepackage{etoolbox}
\makeatletter
\patchcmd\longtable{\par}{\if@noskipsec\mbox{}\fi\par}{}{}
\makeatother
% Allow footnotes in longtable head/foot
\IfFileExists{footnotehyper.sty}{\usepackage{footnotehyper}}{\usepackage{footnote}}
\makesavenoteenv{longtable}
\usepackage{graphicx}
\makeatletter
\def\maxwidth{\ifdim\Gin@nat@width>\linewidth\linewidth\else\Gin@nat@width\fi}
\def\maxheight{\ifdim\Gin@nat@height>\textheight\textheight\else\Gin@nat@height\fi}
\makeatother
% Scale images if necessary, so that they will not overflow the page
% margins by default, and it is still possible to overwrite the defaults
% using explicit options in \includegraphics[width, height, ...]{}
\setkeys{Gin}{width=\maxwidth,height=\maxheight,keepaspectratio}
% Set default figure placement to htbp
\makeatletter
\def\fps@figure{htbp}
\makeatother
\setlength{\emergencystretch}{3em} % prevent overfull lines
\providecommand{\tightlist}{%
  \setlength{\itemsep}{0pt}\setlength{\parskip}{0pt}}
\setcounter{secnumdepth}{5}
\usepackage{booktabs}
\usepackage[T1]{fontenc}
\usepackage{color}
\usepackage{xspace}
\usepackage{tikz-cd}
\usepackage{mathtools}
\usepackage{mathrsfs}
\usepackage{comment}
\usepackage{commath}
\usepackage{pict2e}
\usepackage{float}
\usepackage{array, makecell}
\usepackage{amsthm}														% 
\usepackage{amsmath}
%\usepackage[tagpdf]{axessibility} 
\usepackage[ruled,vlined,shortend]{algorithm2e} 
\usepackage{graphicx}
\usepackage{multicol}
\usepackage{gensymb}
%\usepackage[a-3b]{pdfx}
\graphicspath{ {./images/} }
\usetikzlibrary{shapes.geometric,arrows}
\def\TikZ{Ti\emph{k}Z\ }
\renewcommand{\algorithmcfname}{Algoritmi}
\usepackage{babel}
  \addto{\captionsfinnish}{\renewcommand{\bibname}{Lähteet}}
\usepackage{geometry}
\usepackage{afterpage}
\geometry{
    a4paper,
    total={150mm,237mm},
    left=30mm,
    top=30mm,
    }
\usepackage[numbers]{natbib}
\newcommand{\tekija}{{Lasse Rintakumpu}}
\newcommand{\titteli}{{}} 
\newcommand{\otsikko}{{Hiukkassuodin- ja hiukkassiloitinalgoritmit sekä niiden soveltaminen AoA-menetelmään perustuvassa Bluetooth-sisätilapaikannuksessa}} 
\newcommand{\tutkielma}{{Pro gradu }}
\newcommand{\aika}{{Lokakuu 2024}} 
\newcommand{\paaaine}{{Tilastotiede}} 
\newcommand{\ohjaaja}{{Ohjaajan titteli (Prof./Dos./FT) ja nimi }} %
\newcommand{\tarkastaja}{{Toisen tarkastajan titteli (Prof./Dos./FT) ja nimi}} 
\ifLuaTeX
  \usepackage{selnolig}  % disable illegal ligatures
\fi
\usepackage[]{natbib}
\bibliographystyle{plainnat}
\IfFileExists{bookmark.sty}{\usepackage{bookmark}}{\usepackage{hyperref}}
\IfFileExists{xurl.sty}{\usepackage{xurl}}{} % add URL line breaks if available
\urlstyle{same}
\hypersetup{
  hidelinks,
  pdfcreator={LaTeX via pandoc}}

\author{}
\date{\vspace{-2.5em}}

\begin{document}

\pagenumbering{roman}
\pagestyle{empty}

\begin{center}
\includegraphics[width=10cm]{UTU_logo_FI}
\end{center}

\vspace{3.0cm}
\begin{center}\large
{\sc \otsikko} 
\end{center}

\vspace{0.5cm}
\begin{center}
\titteli \tekija
\end{center}

\vspace{0.5cm}
\begin{center}
\tutkielma -tutkielma\\
\aika
\end{center}

\vspace{2.5cm}
\begin{center}
\begin{tabular}{l}
Tarkastajat:\\
\ohjaaja \\
\tarkastaja
\end{tabular}
\end{center}

\vspace{2.5cm}
\begin{center}
MATEMATIIKAN JA TILASTOTIETEEN LAITOS
\end{center}

\newpage\null

\vspace{22cm}

\noindent Turun yliopiston laatujärjestelmän mukaisesti tämän julkaisun alkuperäisyys on tarkastettu Turnitin OriginalityCheck-järjestelmällä

\cleardoublepage

\noindent
TURUN YLIOPISTO \newline
Matematiikan ja tilastotieteen laitos\newline

\noindent \textsc{\tekija}: \otsikko \newline
\tutkielma-tutkielma, 81 s. \newline
\paaaine \newline
\aika
\par\noindent{\rule{\textwidth}{.2mm}} \newline


\vspace{4mm}\noindent Tutkielmassa esitetään hiukkassuodin- ja hiukkassiloitinalgoritmien teoria Bayesilaisessa tilastotieteellisessä viitekehyksessä. Lisäksi tutkielmassa käsitellään lyhyesti hiukkassuotimien varianssin estimointia.

\vspace{4mm}\noindent Empiirisenä esimerkkinä tutkielmassa tarkastellaan hiukkassuodin- ja hiukkassiloitinalgoritmien käyttöä AoA-teknologiaan perustuvassa Bluetooth-sisätilapaikannusratkaisussa.

\vspace{4mm}\noindent Asiasanat: SMC-menetelmät, Monte Carlo -menetelmät, sekventiaalinen Monte Carlo, suodinongelma, hiukkassuodin, hiukkassiloitin, SIR-algoritmi, sisätilapaikannus, BLE, AoA, triangulaatio, Bayesilainen päättely

\cleardoublepage

\cleardoublepage

\pagestyle{plain} 
\pagenumbering{arabic} 

{
\setcounter{tocdepth}{3}
\tableofcontents
}
\hypertarget{lopuksi}{%
\section{Lopuksi}\label{lopuksi}}

Tässä tutkielmassa on esitetty pääpiirteittäin hiukkassuodin- ja hiukkassiloitinalgoritmien teoria Bayesilaisessa tilastotieteellisessä viitekehyksessä. Tutkielmassa on lisäksi käyty läpi uudelleenotantaa efektiivisen otoskoon perusteella hyödyntävä SIR-suodinalgoritmi sekä käsitelty algoritmin varianssin estimointia. Tutkielmassa on myös esitetty SIR-algoritmin parametrien valintaan, suorituskykyyn sekä konvergenssiin liittyviä tuloksia.

Tutkielmassa on lisäksi esitetty WB-sisätilapaikannusalgoritmi, joka toteuttaa SIR-algoritmin, estimoi jakaumaestimaatin varianssin ja hyödyntää sisätilapaikannuksen karttasovitusalgoritmia. Tutkielmassa on lopuksi tarkasteltu miten eri suunnitteluparametrien valinnat vaikuttavat algoritmin suorituskykyyn kattavan, todelliseen ongelmaan ja dataan perustuvan paikannuskokeen avulla.

Algoritmia ja järjestelmää voitaisiin mahdollisesti edelleen parantaa esimerkiksi käyttämällä tagia, jonka kiihtyvyysmittarin tuottama data olisi AT-2-tagia luotettavampaa. Kiihtyvyyssataa voitaisiin hyödyntää esimerkiksi informatiivisen liikemallin luomisessa.

Koe-esimerkin perusteella tutkielmassa käsitelty ALVar-varianssi estimoi hyvin jakaumaestimaatin varianssia. ALVar-varianssiestimaattia voitaisiin edelleen hyödyntää esimerkiksi adaptiivisen viipeen siloittimen toteuttamisessa. Vastaavasti siloitteluun voitaisiin käyttää yksinkertaisesti XXX TODO KALMAN.

Varianssiestimaattia olisi mahdollista hyödyntää myös WB-sisätilapaikannusalgoritmin dynaamisen mallin kohinaparametrin \(q\) säätämisessä niin, että kohinaa lisättäisiin tai vähennettäisiin varianssiestimaatin perusteella. Tämä voitaisiin toteuttaa esimerkiksi käyttämällä lähtökohtaisesti pienempää \(q\)-arvoa ja lisäämällä siihen keskihajonnan estimaatti, jota on painotettu suunnitteluparametrilla \(\alpha\), ts. ennen dynaamisen mallin sovitusta päivtettäisiin kohinaparametri \(q_k = q_k+\alpha\sqrt{\hat{\sigma}^2_{k-1}}\). Vastaavaa mallia ovat käyttäneet esimerkiksi Xu ja Li (2006) \citep{Xu-2006}.

Varianssiestimaattia voisi hyödyntää myös hiukkasten lukumäärän \(N\) adaptiivisessa valinnassa niin, että matala varianssi vähentää estimoinnissa käytettävien hiukkasten määrää. Tällaista varianssiin perustuvaa hiukkasmäärän muuttamista ehdottavat muun muassa Lee ja Whiteley (2018) \citep{Lee-2018}. Vastaavasti hiukkasten määrää voisi säätää myös arvioimalla hiukkassuotimen konvergenssia esimerkiksi Elviran \&al.~(2017) ehdottamalla menetelmällä \citep{Elvira-2017}. Hiukkasten määrän dynaamisella valinnalla saavutettaisiin todennäköisesti merkittäviä laskennallisia säästöjä, erityisesti koska WB-sisätilapaikannusalgoritmia ajetaan samanaikaisesti useiden satojen AT-2-tägien paikantamiseen.

Algoritmin suorituskykyä voitaisiin mahdollisesti myös parantaa käyttämällä RSSI-kynnysarvon sijaan tai ohella esimerkiksi menetelmää, jossa datasta poistettaisiin kullakin aika-askeleella ne kulmahavainnot, jotka poikkeavat kulmahavaintojen suuntakonsensuksesta (kts. esim. Boquet \&al.~(2024) \citep{Boquet-2024}). Jättämällä osa havainnoista uskottavuusfunktioiden ulkopuolelle nopeuttaisi algoritmin laskentaa, mutta saattaisi parantaa myös algoritmin tarkkuutta.

Liitteenä olevia polkuja tarkastelemalla puolestaan huomataan, että nyt toteutetun algoritmin sijantiestimaatilla on taipumus jäädä osassa testiympäristöä jälkeen itse tagin sijainnista. Tätä ongelmaa voitaisiin mahdollisesti lieventää käyttämällä mm. Yi Chenging \&al.~artikkelissa ``Improved Particle Filter Algorithm for Multi-Target Detection and Tracking'' (2024) \citep{Cheng-2024} esittämää menetelmää, jossa partikkelit jaetaan ns. seurantahiukkasiin sekä etsintähiukkasiin, joista ainoastaan edellisiä käytetään sijaintiestimaatin luomisessa ja jälkimmäisten annetaan liikkua suuremmilla kohina-arvoilla. Näin mahdollistetaan satunnaiskulkumallilla laajempi signaaliavaruuden tutkinta ja nopeampi ongelmatilanteista toipuminen ilman, että sijaintiestimaatit kärsivät liikemalliin lisätystä kohinasta.

Ongelma voitaisiin ratkaista myös käyttämällä ekskluusiopolygonien ohella ns. pehmeitä rajoitteita (kts. esim. Liu \&al.~(2019) \citep{Liu-2019}), jolloin ekskluusiopolygonien ympärille luotaisiin rajoitevyöhyke, jolle osuvat hiukkaset saisivat rangaistuksen riippuen siitä, kuinka lähellä itse ekskluusiopolygonin reunaa ne ovat. Rangaistusarvon luomisessa voitaisiin käyttää esimerkiksi eksponentiaalista tai kvadraattista hajoamista.

Algoritmin optimaalisia suunnitteluparametreja etsivän kokeen voisi myös toteuttaa käytetyn yksinkertaisen ruudukkohaun sijaan esimerkiksi geneettisillä algoritmeilla, Bayesilaisella optimoinnilla tai sensitiivisyysanalyysilla, olettaen, että parametrit ovat toisistaan riippumattomia.

\end{document}
