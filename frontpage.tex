\pagenumbering{roman}
\pagestyle{empty}

\begin{center}
\includegraphics[width=10cm]{UTU_logo_FI}
\end{center}

\vspace{3.0cm}
\begin{center}\large
{\sc \otsikko} 
\end{center}

\vspace{0.5cm}
\begin{center}
\titteli \tekija
\end{center}

\vspace{0.5cm}
\begin{center}
\tutkielma -tutkielma\\
\aika
\end{center}

\vspace{2.5cm}
\begin{center}
\begin{tabular}{l}
Tarkastajat:\\
\ohjaaja \\
\tarkastaja
\end{tabular}
\end{center}

\vspace{2.5cm}
\begin{center}
MATEMATIIKAN JA TILASTOTIETEEN LAITOS
\end{center}

\newpage\null

\vspace{22cm}

\noindent Turun yliopiston laatujärjestelmän mukaisesti tämän julkaisun alkuperäisyys on tarkastettu Turnitin OriginalityCheck-järjestelmällä

\cleardoublepage

\noindent
TURUN YLIOPISTO \newline
Matematiikan ja tilastotieteen laitos\newline

\noindent \textsc{\tekija}: \otsikko \newline
\tutkielma-tutkielma, 81 s. \newline
\paaaine \newline
\aika
\par\noindent{\rule{\textwidth}{.2mm}} \newline


\vspace{4mm}\noindent Tutkielmassa esitetään hiukkassuodin- ja hiukkassiloitinalgoritmien teoria Bayesilaisessa tilastotieteellisessä viitekehyksessä. Lisäksi tutkielmassa käsitellään lyhyesti hiukkassuotimien varianssin estimointia.

\vspace{4mm}\noindent Empiirisenä esimerkkinä tutkielmassa tarkastellaan hiukkassuodin- ja hiukkassiloitinalgoritmien käyttöä AoA-teknologiaan perustuvassa Bluetooth-sisätilapaikannusratkaisussa.

\vspace{4mm}\noindent Asiasanat: SMC-menetelmät, Monte Carlo -menetelmät, sekventiaalinen Monte Carlo, suodinongelma, hiukkassuodin, hiukkassiloitin, SIR-algoritmi, sisätilapaikannus, BLE, AoA, triangulaatio, Bayesilainen päättely

\cleardoublepage

\cleardoublepage

\pagestyle{plain} 
\pagenumbering{arabic} 